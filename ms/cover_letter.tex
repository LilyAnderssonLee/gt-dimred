\documentclass{article}
\setlength\parindent{0pt}

\begin{document}
Dear Editor,
\\\\
We are pleased to submit our manuscript ``UMAP reveals cryptic population structure and phenotype heterogeneity in large genomic cohorts'' for publication in eLife. This work applies a newly developed dimensional reduction method called UMAP to visualize population structure in genomic datasets.
\\\\
By applying UMAP to large contemporary human genetic datasets (the 1000 Genomes Project, the Health and Retirement Study, and the UK biobank), we reveal a wide range of patterns that had never been observed in these well-studied cohorts. For example, we identify a cluster of Hispanic individuals that is largely restricted to the US Mountain region, who are likely descendants of the Hispano population. We also identify multiple associations between genetics, geography, and phenotypes in the UK biobank and sub-continental population structure within 1000 Genomes Project populations.
\\\\
This work has broad applicability. Our approach has already been implemented as a visualization tool by leading cohorts such as GnomAD\footnote{Fancioli, Laurent. ``gnomAD v2.1''. https://macarthurlab.org/2018/10/17/gnomad-v2-1/ (2018-10-17).}, and used as a preprocessing step for machine learning approaches\footnote{Tonkin-Hill, Gerry, et al. ``Fast Hierarchical Bayesian Analysis of Population Structure.'' bioRxiv (2018): 454355.}. Our preprint is also in the 99th percentile of all-time bioRxiv preprints for the Altmetric impact statistic. Because of its extreme computational tractability and its ability to efficiently summarize genomic information, we expect that our approach will become a new standard tool in data analysis for large genomic studies. 
\\\\
The manuscript presents new methodology and applications that reveal fine-scale genomic and phenotypic variation in large-scale cohorts.  We believe that this will be of interest to a broad variety of disciplines including population genetics, medical genetics, anthropology, and data science, which makes the manuscript an ideal candidate for eLife.
\\
\\
Sincerely,
\\
Simon Gravel and Alex Diaz-Papkovich
\\
on behalf of all authors.

\end{document}
\documentclass{article}
\setlength\parindent{0pt}

\begin{document}
Dear Editor,
\\\\
We are pleased to resubmit our manuscript \emph{UMAP reveals cryptic population structure and phenotype heterogeneity in large genomic cohorts} for publication in PLOS Genetics. Since our previous submission we have thoroughly edited the manuscript to address reviewer and preprint reader comments in detail, to provide guidance on usage and interpretation of UMAP, and to provide rigorous follow up on some findings.
\\\\
Among the findings that are now discussed in more detail, we found differences between East Asian populations in the distributions of FEV1, a clinically important measurement of lung function, which have not been identified to date to the best of our knowledge. We have also identified more detailed population structure patterns using the country of birth for individuals in the UK biobank and demonstrated how to use UMAP to identify patterns in ancestry on a global scale.
\\\\
The methods described in our preprint continue to be used by leading groups including the National Geographic Genographic Project\footnote{Dai, Chengzhen L., et al. "Population histories of the United States revealed through fine-scale migration and haplotype analysis." BioRxiv (2019): 577411}, GnomAD \footnote{Karczewski, Konrad J., et al., BioRxiv (2019): 531210 }, and 23andMe\footnote{23andMe. https://blog.23andme.com/ancestry/23andme-tests-new-ancestry-breakdown-in-central-and-south-asia/. ``23andMe Tests New Ancestry Breakdown in Central and South Asia'' (2019-04-03)}. 
\\
We therefore believe that the manuscript will be of interest to the wide readership of PLoS Genetics and hope that you will find this revision suitable for publication.
\\
Sincerely,
\\
Alex Diaz-Papkovich and Simon Gravel
\\
on behalf of all authors.

\end{document}